\documentclass{article}
\usepackage{xeCJK}
\usepackage{fvextra}
\usepackage{graphicx}
\graphicspath{{./picture/}}  
\usepackage{fontspec}       % 设置字体
\usepackage{titlesec}       % 调整标题格式
\setlength{\parindent}{2em}

\title{{\LARGE \textbf{系统开发工具基础(第一次实验报告)}}} % 标题:加粗 + 大号
\author{{\Large 程乾楠 24020007014}} % 作者:大号
\date{\today}

\begin{document}
\fvset{breaklines=true}
\begin{titlepage}
\maketitle
\end{titlepage}

\section*{1.关于Latex的使用}
\subsection*{1.0Latex的界面的介绍}
overleaf可以提供网页端的Latex编辑器的使用,本实验就是在overleaf的帮助下完成的,欢迎提出指正。
\subsection*{1.1Latex中的转义}
\begin{Verbatim}
在使用Latex进行编辑时,\documentclass{article}这一行代码是声明文档的类型,在”\begin{document}“和”\end{document}“之间的就是文档的内容。如果想要使文本内容出现一些代码的文本,就可以使用转义。常见的转义方法有\begin{verbatim},end{verbatim}。\texttt{} 包裹命令,例如\texttt{\textbackslash documentclass\{article\}}。
\end{Verbatim}

\subsection*{1.2Latex中如何使用中文}
要在声明文档类型后使用xeCJK这个包,使用的代码如下:
\begin{Verbatim}
\usepackage{xeCJK}
\end{Verbatim}
而且要在Menu中改变编译器为XeLatex,然后就可以使用中文了。

\subsection*{1.3换行符的使用}
在Latex中,换行操作可以使用:
\begin{Verbatim}
\newline或者\\
\end{Verbatim}
它们不会触发缩进,但是使用两段文字中间空一行的操作,那样就会判定为两段,会触发缩进。

\subsection*{1.4换页符的使用}
在Latex中,换页符可以使用:
\begin{Verbatim}
\newpage,功能是从当前位置强制开始新的一页。而\clearpage,但是后者除了换页,还会把所有未排版的浮动体(figure/table)先输出。
\end{Verbatim}

\subsection*{1.5居中对齐}
在写左右居中对齐时,可以用:
居中对齐:
\begin{Verbatim}
将想要居中对齐的内容放到:
\begin{center} 和\end{center} 中
\end{Verbatim}
\newpage

\subsection*{1.6居左居右对齐}
同1.5一样,将center换为flushright,flushleft就可以了。

\subsection*{1.7插入图片(固定位置)}
在Latex固定位置中插入图片:
\begin{Verbatim}
在固定位置插入图片,需要先新建文件夹存放图片,然后上传图片,需要加一个\usepackage{graphicx}
之后设置图片搜索路径:
比如\graphicspath{{./picture/}} ,graphicspath 可以指定图片文件夹。
然后写入以下代码:
\includegraphics[width=0.6\textwidth]{图片的文件名}
或者\includegraphics[width=n cm,height=m cm]{图片的文件名}
n和m为小于页面的宽度和长度的数字。
完成这些就可以了。
展示效果:
\end{Verbatim}
\includegraphics[width=0.5\textwidth]{9FE6F6EC3F0468CC7A72ACFB940780F5}
\includegraphics[width=3cm,height=2cm]{9FE6F6EC3F0468CC7A72ACFB940780F5}\newpage

\subsection*{1.8插入图片(浮动位置)}
在Latex浮动位置中插入图片:
\begin{Verbatim}
在以下代码的基础上:
\includegraphics[width=0.6\textwidth]{图片的文件名}
前后加上\begin{figure}和\end{figure}就可以了,其代表含义为图片出现的位置不一定在代码出现的位置上,而是在这个位置附近浮动
\end{Verbatim}
\subsection*{1.9英文字体改变}
\begin{Verbatim}
使用Hello,world作为例子:
\textbf{Hello,world}\par
\underline{Hello, world}\par
\textit{Hello, world}\par
{\small Hello, world}\par
{\large Hello, world}\par
{\Large Hello, world}\par
{\LARGE Hello, world}\par
{\huge Hello, world}\par
{\Huge Hello, world}\par
\end{Verbatim}
\begin{center}
\textbf{Hello,world}\par
\underline{Hello, world}\par
\textit{Hello, world}\par
{\small Hello, world}\par
{\large Hello, world}\par
{\Large Hello, world}\par
{\LARGE Hello, world}\par
{\huge Hello, world}\par
{\Huge Hello, world}\par
\end{center}

\subsection*{1.10中文字体改变}
\begin{center}
\textbf{你好,世界} \par        % 中文粗体
\underline{你好,世界} \par      % 中文下划线
\textit{你好,世界} \par         % 中文斜体
{\small 你好,世界} \par        % 小号字体
{\large 你好,世界} \par        % 大号字体
{\Large 你好,世界} \par        % 更大字体
{\LARGE 你好,世界} \par        % 进一步增大字体
{\huge 你好,世界} \par         % 很大的字体
{\Huge 你好,世界} \par\newpage         % 最大字体
\end{center}

\subsection*{1.11插入表格}
使用:
\begin{table}[htbp]   % 'htbp'表示浮动位置,确保表格尽量在合适的位置
    \centering
    \begin{tabular}{|c|c|c|}   % 表格有三列,每列都居中对齐,边界用竖线表示
        \hline  % 添加表格的横线
        列1 & 列2 & 列3 \\   % 表格第一行
        \hline
        内容1 & 内容2 & 内容3 \\
        \hline
        内容4 & 内容5 & 内容6 \\
        \hline
    \end{tabular}
    \caption{简单表格示例}  % 表格标题
    \label{tab:simple_table} % 表格标签,便于引用
\end{table}
\subsection*{1.12图片加标题}
\begin{figure}[h]
\caption{鸣潮角色图片}
\centering
\includegraphics[width=0.4\textwidth]{9FE6F6EC3F0468CC7A72ACFB940780F5}
\end{figure}

\subsection*{1.12图片引用}
\begin{figure}[h]
\caption{鸣潮角色图片}
\centering
\includegraphics[width=0.4\textwidth]{9FE6F6EC3F0468CC7A72ACFB940780F5}
\label{fig:sample_image} 
\end{figure}
如图 \ref{fig:sample_image} 所示,这是一张示例图片。

\section*{2.git的使用}
\subsection*{2.1在VS Code 里配置 Git 的用户名和邮箱,让它和 GitHub 对应上}
git config --global user.name "你的GitHub用户名"
git config --global user.email "你的GitHub邮箱"
使用这两行指令,可以用VS Code 里配置 Git 的用户名和邮箱,user.name 建议和 GitHub 上的用户名一致。
user.email 必须是 GitHub 账号绑定的邮箱,否则 GitHub 可能不会显示你是这次提交的作者。
\subsection*{2.2使用SSH key 的方式把本地 Git 和 GitHub 连接起来}
通过git bash输入这行指令ssh-keygen -t rsa,之后会显示储存位置,
找到公钥,并把它输入到github上的SSH keys里,名字可以随便取。

\includegraphics[width=0.6\textwidth]{ad}
这里是弄好的截图。

\subsection{2.3给 Git 设置代理}
通过以下两行指令设置代理:
git config --global http.proxy http://127.0.0.1:代理的端口
git config --global https.proxy http://127.0.0.1:代理的端口

\subsection{2.4将GitHub上的仓库克隆到本地文件夹中}
使用HTTPS方式克隆:先打开想要克隆到的文件夹,在里面打开git bash,然后输入:
git clone https://github.com/yourname/myrepo.git
\subsection*{2.5查看当前文件状态}
通过git status可以完成。

\includegraphics[width=0.6\textwidth]{ams}

\subsection*{2.6将文件上传到github}
先添加到暂存区,通过git add filename.后缀 来进行添加。

\includegraphics[width=0.6\textwidth]{qqq}

再提交到本地仓库:
git commit -m "本地仓库名“

然后通过HTTPS连上远程仓库:
本人最开始把通过SSH连接远程仓库的地址格式弄错了,弄成这样了:

git@https://github.com/target107/git666.git

导致我后来又修改了origin指向的地址,
git remote set-url origin https://github.com/target107/git666.git

之后才推送成功的:
git push -u origin master\newpage

\includegraphics[width=0.7\textwidth]{www}

\includegraphics[width=0.7\textwidth]{eee}

\section*{3.实验后的反思}

\subsection*{3.1实验中存在的问题}
在使用转义时,没有考虑到verbatim不会换行的问题,使用ai提出的方法时,没有考虑到自动换行是加强内容,要使用fvextra包。
在学习git时遇到的问题很多,什么操作也不会,也不知道是干什么的,后来的学习才认识到git的作用,就比如SSH连接远程仓库,最后发现密钥用不了,试验结束后再看看能不能用SSH连接到github仓库。

\subsection*{3.2实验后的反思}
在学习时,一定要举一反三,学习时要灵活多变,边学边实践才是最正确的学习方法
\end{document}